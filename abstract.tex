\begin{abstract}
Mobile apps often require access to private data, such as the device ID or location. At the same time, popular platforms like Android and iOS have limited support for user privacy. This frequently leads to unauthorized disclosure of private information by mobile apps, e.g. for advertising and analytics purposes.
%
This paper addresses the problem of privacy enforcement in mobile systems, which we formulate as a classification problem: When arriving at a privacy sink (e.g., database update or outgoing web message), the runtime system must classify the sink's behavior as either legitimate or illegitimate. The traditional approach of information-flow (or taint) tracking applies ``binary'' classification, whereby information release is legitimate iff there is no data flow from a privacy source to sink arguments. While this is a useful heuristic, it also leads to false alarms.

We propose to address privacy enforcement as a learning problem, relaxing binary judgments into a quantitative/probabilistic mode of reasoning. Specifically, we propose a Bayesian notion of statistical classification, which conditions the judgment whether a release point is legitimate on the evidence arising at that point. In our concrete approach, implemented as the \Tool\ system which is soon to be featured in a commercial product, the evidence refers to the similarity between the data values about to be released and the private data stored on the device. Compared to TaintDroid, a state-of-the-art taint-based tool for privacy enforcement, \Tool\ is substantially more accurate. Applied to 54 top-popular Google Play apps, \Tool\ is able to detect 64 privacy violations with only 1 false alarm. 
\end{abstract}